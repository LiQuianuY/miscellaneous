\documentclass{article}

\usepackage{amsmath}
\usepackage{amsthm}
\usepackage{enumerate}

\newtheorem{property}{Property}[section]
\newtheorem{theorem}{Theorem}[section]
\newtheorem{corollary}{Corollary}[theorem]
\newtheorem{lemma}[theorem]{Lemma}
\theoremstyle{definition}
\newtheorem{definition}{Definition}[section]
\newtheorem{example}{Example}[section]
\theoremstyle{remark}
\newtheorem*{remark}{Remark}
\newtheorem*{convention}{Convention}



\title{Notes to Analysis by Amann}
\date{\today}
\author{Lio Quid}

\begin{document}
\maketitle
\tableofcontents
\newpage 

\section{Fundamentals of Logic}
In logic of mathematics we are concerned with \textbf{statements} that are claimed to be true or false.
That is, each statement has the \textit{truth value} 'true'(T) or 'false'(F),
and there are no other possibilities.

\subsection{Exercises}
1. ``The Simpsons are coming to visit this evening,'' announced Maud Flanders.
``The whole family — Homer, Marge and their three kids, Bart, Lisa and Maggie?'' asked Ned Flanders dismayed. 
Maud, who never misses a chance to stimulate her husband's logical thinking, replied, 
``I'll explain it this way: 
If Homer comes then he will bring Marge too.
At least one of the two children, Maggie and Lisa, are coming. 
Either Marge or Bart is coming, but not both. 
Either both Bart and Lisa are coming or neither is coming. 
And if Maggie comes, then Lisa and Homer are coming too. 
So now you know who is visiting this evening.''

Who is coming to visit?

\begin{proof}
    Let $H$, $M_1$, $B$, $L$, $M_2$ be statements defined in the following:
    \begin{align*}
        H &:= \text{Homer is coming to visit.}\\
        M_1 &:= \text{Marge is coming to visit.}\\
        B &:= \text{Bart is coming to visit.}\\
        L &:= \text{Lisa is coming to visit.}\\
        M_2 &:= \text{Maggie is coming to visit.}
    \end{align*}
    Then according to the definitions made above,
    we could rewrite the hypothesis in a cleaner way.
    The hypothesis says the following statements are true:
    \begin{enumerate}
        \item $H \Rightarrow M_1$.
        \item $L \lor M_2$.
        \item $(M_1 \land \neg B) \lor (\neg M_1 \land B)$.
        \item $(B \land L) \lor (\neg B \land \neg L)$.
        \item $M_2 \Rightarrow (L \land H)$.
    \end{enumerate}
    We start from one of two implications above, 
    say, we assume $M_2$ is true, i.e., Maggie is coming.
    Then by statement (5) we have $L \land H$ be true,
    thus we know both $L$ and $H$ must be true.
    Then from statement (4) we know that $B$ must be true,
    and from statement (1) that $M_1$ must be true.
    Finally, from statement (3) we know that $B$ must be false, but this is a contradiction. (earlier we have just asserted that $B$ is true)
    Hence we know that $M_2$ must be false. (Otherwise contradiction would happen)

    Again, since $M_2$ is false, we know that $L$ must be true by statement (2).
    Then we know that $B$ must be true from statement (4) and then $M_1$ must be false from statement (3).
    Since $M_1$ is false, we know that $H$ must be false because statement (1) is true.
    Now it seems like no logical contradiction occur and everyone in the simpsons are happy.

    In conclusion, only statements $L$ and $B$ are true, that is, Lisa and Bart are coming to visit.
\end{proof}

2. In the library of Count Dracula no two books contain exactly the same number of words.
The number of books is greater than the total number of words in all the books.
These statements suffice to determine the content of at least one book in Count Dracula's
library. What is in this book?
\end{document}